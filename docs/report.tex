\documentclass[a4paper,12pt]{article}
\usepackage[utf8]{inputenc}
\usepackage[T1]{fontenc}
\usepackage{lmodern}
\usepackage{geometry}
\usepackage{amsmath}
\usepackage{graphicx}
\usepackage{hyperref}
\usepackage{enumitem}
\usepackage{xcolor}

\geometry{margin=1in}

\title{Relatório: Trabalho de Banco de Dados}
\author{Natanael Santos da Silva}
\date{12/05/2025}

\begin{document}

\maketitle

\section{Introdução}
Este relatório apresenta a implementação de um projeto de banco de dados com SQLite, incluindo a criação de tabelas, triggers, views e testes automatizados com pytest. O projeto segue as especificações fornecidas, com ênfase em boas práticas e organização.

\section{Conceitos Fundamentais}

\subsection{Integridade Referencial}
A integridade referencial garante que as chaves estrangeiras em uma tabela correspondam a chaves primárias válidas em outra tabela, evitando registros órfãos. Por exemplo, na tabela \texttt{orders}, o campo \texttt{user\_id} referencia \texttt{id} da tabela \texttt{users}.

\subsection{View}
Uma view é uma consulta armazenada que atua como uma tabela virtual. No projeto, a view \texttt{user\_orders} combina dados de \texttt{users} e \texttt{orders} para facilitar consultas.

\subsection{Trigger}
Triggers são ações automáticas disparadas por eventos no banco, como inserções. A trigger \texttt{order\_insert} registra logs automaticamente ao inserir pedidos.

\subsection{Join}
Um join combina dados de múltiplas tabelas com base em condições. O projeto usa INNER JOIN na view \texttt{user\_orders} para relacionar \texttt{users} e \texttt{orders}.

\subsection{Teste de Performance}
Testes de performance verificam a eficiência de operações, como o uso de índices para acelerar buscas. O teste \texttt{test\_index\_performance} confirma que o índice \texttt{idx\_name} é utilizado.

\section{Estudo de Caso}
O projeto implementa um sistema de gerenciamento de pedidos com as seguintes funcionalidades:
\begin{itemize}
    \item \textbf{Tabelas}: \texttt{users}, \texttt{orders}, \texttt{logs}.
    \item \textbf{Trigger}: Registra inserções em \texttt{orders}.
    \item \textbf{View}: Exibe usuários e seus pedidos.
    \item \textbf{Testes}: Valida atualização, view, trigger, chave estrangeira, índice e operações adicionais.
\end{itemize}

\section{Testes Implementados}
Os testes estão organizados em arquivos separados na pasta \texttt{tests/}:
\begin{enumerate}
    \item \texttt{test\_update.py}: Testa atualização de usuário.
    \item \texttt{test\_view.py}: Testa a view \texttt{user\_orders}.
    \item \texttt{test\_trigger.py}: Testa a trigger \texttt{order\_insert}.
    \item \texttt{test\_foreign\_key.py}: Testa integridade referencial.
    \item \texttt{test\_index.py}: Testa o uso de índice.
    \item \texttt{test\_additional.py}: Testa inserção, deleção, join, transação e concorrência.
\end{enumerate}

\end{document}